\documentclass[11pt]{article}

\usepackage{Sweave}
\begin{document}
\Sconcordance{concordance:lmPackage.tex:lmPackage.Rnw:%
1 2 1 1 0 26 1 1 2 4 0 1 2 2 1 1 2 4 0 1 2 5 1 1 2 4 0 1 2 4 1 1 2 1 0 %
1 1 6 0 1 2 3 1 1 2 1 0 1 1 3 0 1 2 2 1}


\title{\bf Ejemplo creaci\'on de una vignette: modelo lineal}
\vspace{1cm}
\author{Juan R Gonz\'alez}

\maketitle

\begin{center}
  Instituto de Salud Global Barcelona (ISGlobal)
  \vspace{1cm}
  {\tt juan.rgonzalez@isglobal.org}
  {\tt http://www.creal.cat/jrgonzalez/software.htm}
\end{center}

\tableofcontents

%%%%%%%%%%%%%%%%%%%%%%%%%%%%%%%%%%%%%%%%%%%%%%%%%%%%%%%%%%
\section{Introduction}

Este documento da una visi\'on general del paquete {\tt lmPackage} que est\'a
accesible en CRAN.  Mediante esta librer\'ia se puede llevar a cabo ...

\section{Getting started}
\noindent El paquete {\tt lmPackage} utiliza la librer\'a {\tt survival}.  Empezamos cargando las librer\'ias necesarias.

\begin{Schunk}
\begin{Sinput}
> library(survival)
\end{Sinput}
\end{Schunk}

\noindent Despues, la librer\'ia puede cargarse ejecutando

\begin{Schunk}
\begin{Sinput}
> library(lmPackage)
\end{Sinput}
\end{Schunk}


\section{The data}

\noindent Los datos pueden importarse usando la funci\'on {\tt read.delim} de forma usual. El paquete tiene unos datos (sobre unos lagos) que pueden usarse como ejemplo. Se cargan usando  la funci\'on data

\begin{Schunk}
\begin{Sinput}
> bass <- read.delim("../extdata/bass.txt")
\end{Sinput}
\end{Schunk}

\section{Model parameter estimates}

\noindent El modelo se puede estimar usando la funci\'on {\tt lmMod}

\begin{Schunk}
\begin{Sinput}
> mod <- lmModFinal(Mercury ~ Alkalinity, data=bass)
> names(mod)
\end{Sinput}
\begin{Soutput}
[1] "coefficients" "vcov"         "sigma"        "df"           "data"        
\end{Soutput}
\end{Schunk}


\noindent Se puede crear una figura con el modelo mediante la siguiente instrucci\'on:

\begin{Schunk}
\begin{Sinput}
> plot(mod$data[,2], mod$data[,3], ylab="Mercury", xlab="Alkalinity")
> abline(mod)
\end{Sinput}
\end{Schunk}


\end{document}
